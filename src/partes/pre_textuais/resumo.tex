\setlength{\absparsep}{18pt} % ajusta o espaçamento dos parágrafos do resumo
\begin{resumo}
	Carros autonomos v{\^e}m sendo um campo de estudo cada vez mais discutido
	nos {\'u}ltimos anos, com um mercado em forte crescimento - tendo sua taxa
	de crescimento acumulativa estipulada em 25,7\% \cite{globe2022}. Mesmo com
	esse crescimento, o mercado se v{\^e} restrito a poucas empresas, que
	est{\~a}o dispostas a arcar com os grandes investimento que o
	desenvolvimento e treinamento dessas solu{\c c}{\~o}es podem trazer. Com
	base nessa situa{\c c}{\~a}o, este projeto visa explorar uma alternativa ao
	modelo tradicional de carro aut{\^o}nomo, deixando de usar GPS e vis{\~a}o
	computacional e passando a se basear em aprendizado por refor{\c c}o, e
	principalmente no algoritmo NEAT (\textit{Neuroevolution of Augmenting
	Topologies}), um algoritmo baseado em redes neurais e algoritmo
	gen{\'e}tico para o aprendizado com m{\'i}nima necessidade de recursos -
	tanto em termos de aquisi{\c c}{\~a}o de dados quanto de performance. Foi
	utilizado um ambiente simulado em que o NEAT utiliza sensores de
	dist{\^a}ncia para progressivamente aprender a percorrer o caminho
	designado. Enquanto tenha sido obtido sucesso neste aprendizado, a
	inconsist{\^e}ncia na generaliza{\c c}{\~a}o da solu{\c c}{\~a}o encontrada
	pelo algoritmo (indicada pela performance na troca do percurso ap{\'o}s o
	treinamento) sugere a inaptid{\~a}o deste algoritmo a operar em ambientes
	imprevis{\'i}veis. Mesmo assim, a facilidade de treinamento e funcionamento
	fazem com que a utiliza{\c c}{\~a}o do NEAT seja sujeita a considera{\c
	c}{\~a}o em contextos de ambientes mais controlados e recursos limitados.

	\textbf{Palavras-chave}: Intelig{\^e}ncia Artificial. Aprendizado por Refor{\c c}o. Carros aut{\^o}nomos. Aprendizado de M{\'a}quina.
\end{resumo}

\begin{resumo}[Abstract]
	\begin{otherlanguage*}{english}
		Autonomous cars have been an increasingly discussed field of study in
		the last years, with a strong market growth - it's cummulative annual
		growth rate stipulated in 25.7\% \cite{globe2022}. Even with this
		growth, the market is restricted to few companies that are willing to
		bear the great investments that the development and trainings of these
		solutions can bring. Based on this, this projects seeks to explore an
		alternative to the tradicional autonomous car model, exchanging the use
		of GPS and computer vision with having a base on reinforcement
		learning, and primarily in the algorithm NEAT (\textit{Neuroevolution
		of Augmenting Topologies}), an algorithm based on neural networks and
		genetic algorithm for learning with minimum resources needed - both in
		monetary terms and in performance terms. A simulated environment was
		used in which NEAT uses distance sensors to progressively learn to
		travel the designaded path. While success was achieved in this
		training, the inconsistence of the generalization in the solution found
		by the algorithm (indicated by the performance when the track is
		switched after the training) suggests the inaptitude of the algorithm
		when operating in unpredictable environments. Nevertheless, the ease of
		training and operation make NEAT worth consideration in more
		predictable and low-resource contexts.

        \vspace{\onelineskip}

        \noindent 
        \textbf{Keywords}: Artificial Intelligence. Reinforcement Learning. Autonomous Cars. Machine Learning.
    \end{otherlanguage*}
\end{resumo}
