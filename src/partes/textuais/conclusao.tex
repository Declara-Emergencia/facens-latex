\chapter{Considerações finais}

Apesar da inconsist{\^e}ncia da transi{\c
c}{\~a}o de treino para teste indicar a falta de
generaliza{\c c}{\~a}o do NEAT, e assim a falta
de aptid{\~a}o para uma aplica{\c c}{\~a}o em um
ve{\'i}culo f{\'i}sico, a facilidade e rapidez
de treinamento — pelas possibilidades de
paraleliza{\c c}{\~a}o dadas pelo algoritmo — faz-se 
plaus{\'i}vel o treinamento em cada
ambiente espec{\'i}fico. Por{\'e}m, neste caso
seria dif{\'i}cil justificar o aumento de
complexidade em compara{\c c}{\~a}o a, por
exemplo, um AGV ou at{\'e} mesmo um simples
sistema especialista como um rob{\^o} guiado
por uma fita no ch{\~a}o.

Al{\'e}m disso, pelo método de recompensa utilizado, 
um problema recorrente foi a t{\'e}cnica dos carros de 
ignorar o caminho e andar em c{\'i}rculos — assim aumentando 
a dist{\^a}ncia percorrida e evitando o perigo de colis{\~a}o. 
Isso foi remediado pela presen{\c c}a das recompensas extras 
dispostas no caminho, por{\'e}m continuou sendo algo poss{\'i}vel no 
processo de aprendizado. Um diferente m{\'e}todo de recompensa ou 
regras mais restritas sobre o funcionamento dos carros poderia 
amenizar esses problemas, consequentemente aumentando a complexidade 
por algo que n{\~a} {\'e} preciso ser lidado em outras op{\c c}{\~o}es de algoritmos.

Outra oportunidade de melhoria seriam mais entradas de dados na 
rede para aux{\'i}lio na tomada de decis{\~a}o. Eventualmente, os 
carros encontraram um modo de "\textit{desacelerar}"  mesmo com a 
acelera{\c c}{\~a}o constante da simula{\c c}{\~a}o: virando para 
os dois lados rapidamente, reduzindo a velocidade para frente. 
Enquanto esse m{\'e}todo funciona, adicionar o par{\^a}metro 
de velocidade como uma entrada na rede poderia auxiliar na decis{\~a}o 
de virar ou seguir em frente. No entanto, remover a acelera{\c c}{\~a}o 
constante pode diminuir o incentivo de manter o movimento, 
entrando no ponto anterior de priorizar sobreviv{\^e}ncia a recompensas.

Tendo isso, com o processo de treinamento uma boa
performance de aprendizado comparado com outros
m{\'e}todos semelhantes e custos computacionais 
modestos, junto com uma implementa{\c c}{\~a}o 
relativamente simples (juntando conceitos
básicos e provados na {\'a}rea, Redes Neurais e
Algor{\'i}tmo Gen{\'e}tico), o NEAT pode ser uma
alternativa que vale a pena a ser explorada em
problemas em que generalidade {\'e} desejada
por{\'e}m, n{\~a}o estritamente necess{\'a}ria.
