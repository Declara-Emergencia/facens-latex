\chapter{Introdução}
% ----------------------------------------------------------

De forma geral, esta monografia visa apresentar a implementação de um
ambiente simulado capaz de treinar um veículo autônomo que, através dos
casos de teste, consegue se adaptar ao ambiente em que se situa buscando
chegar a determinado objetivo, através de um algoritmo de neuroevolução que
possui uma topologia aumentante. Desta forma, simulações de possíveis trajetos
e adaptações de percurso podem ser rapidamente criadas e testadas sem a
necessidade de interação com um ambiente físico.

Tratando-se de veículos autônomos, é possível citar os carros sem
motorista, segundo \citeonline{arruda2015}, são veículos capazes de realizar tarefas
[...] sem a intervenção de operadores humanos, sua demanda vem crescendo
ultimamente por conta da acessibilidade que fornecem e por permitirem uma
produtividade maior ao motorista. Considerados também veículos autônomos,
existem aqueles voltados para sistemas logísticos, onde, através de mínima
interação com operadores, salvo manutenções e reprogramações, são capazes
de realizar transporte de objetos por uma linha de produção.

A solução geralmente aplicada para carros autônomos se fundamenta no
princípio da visão computacional. Esta técnica se baseia em utilizar de câmeras
para a captura de imagens que virão a servir de entrada de dados, para que o
algoritmo de direção tome as devidas decisões quanto à melhor decisão a ser
tomada para aquela situação.

O problema desta técnica se encontra justamente na lentidão de colocar
os sistemas em prática, pois é necessário que se criem diversas situações de
teste com muitas variáveis aplicáveis, como iluminação ou resolução da foto, e
considerando que a aplicação é direcionada a carros que navegarão pelas ruas,
é necessário que estes testes sejam prolongados ainda mais para que se haja
uma garantia de segurança a todos ao redor do veículo.

Ao observar veículos autônomos dentro do ambiente fabril, é possível
encontrar máquinas chamadas de Veículos Automaticamente Guiados (AGVs),
seu uso para o transporte de contêineres data de 1993 em um terminal de
contêineres de Rotterdan \cite{saputra2021}. Estes possuem a finalidade
de transportar materiais e produtos pela fábrica, através de sensores, são
capazes de identificar a distância necessária para a movimentação do garfo
quando aplicados a uma empilhadeira ou a distância de segurança para evitar
acidentes. A sua trajetória desses veículos é geralmente predefinida, baseando-
se na planta da fábrica e o caminho que o veículo deve percorrer, como através
de faixas magnéticas, por exemplo. Sendo assim, estes dispositivos falham
quando encontram obstáculos em seu caminho, como um pallet ou caixa que
manterão sua movimentação impedida.

Tratando-se de aprendizado de máquina, existem diversas técnicas que
podem ser empregadas para se chegar a resultados simulados, estas, focadas
na modelagem da estrutura de aprendizado podem ser divididas entre,
supervisionado, não supervisionado e por reforço. A aplicação de cada método
deve ser indicada de acordo com a necessidade da aplicação, e, analisando a
proposta do projeto presente nesta monografia, é possível analisar qual solução
possui a aplicação mais justificável. Em uma observação imediata, o
aprendizado por máquina supervisionado (aquele em que se sabe quais são as
possíveis entradas e saídas equivalentes a elas) se torna inviável, tal que a
aplicação exige que o algoritmo calcule em tempo real a intermissão de possíveis
objetos em seu trajeto, sendo assim impossível definir uma saída previamente.

Este projeto procura explorar um tipo diferente de aprendizado, chamado
de aprendizado por reforço. Esse tipo de aprendizado tem a característica de
não necessitar de supervisão para o treinamento do modelo. Ou seja, não se faz
necessário treinamento em campo ou uma requisição de grande quantidade de
dados para a etapa de treinamento. Como entradas para o modelo serão
utilizados sensores de distância, em vez dos habituais sistemas de visão
computacional. O experimento será feito dentro de um ambiente virtual, com a
representação do carro e trajetos gerados manualmente para treinamento.

O método de aprendizado por reforço faz necessário menos investimentos
antecipados, pois os dados de treinamento serão obtidos por um quociente de
performance ao invés de dados rotulados para retroalimentação. Além disso, a
utilização de sensores diminui significativamente o requerimento de poder
computacional para o sistema comparado com processamento de imagens, e o
mesmo pode ser dito quanto ao algoritmo de aprendizado na etapa de testes.

O aprendizado se dará pelo algoritmo NEAT (\textit{Neuroevolution of
Augmenting Topologies}). Este algoritmo se inicia com uma rede neural com
apenas as entradas e saídas definidas. Por meio de um Algoritmo Genético, o
algoritmo evoluirá a topologia da rede neural de forma a melhorar o desempenho
do carro (neste caso, conseguir dirigir por mais tempo sem colisões).

Exemplificando, a princípio o algoritmo gera diversos carros com topologias
de estrutura mínima para sua rede neural. Os carros que percorrerem a maior 
distância sem colisões serão selecionados para reprodução — seus genes serão 
combinados com outros carros de performance alta, além de também ocorrerem mutações
para o descobrimento de novos comportamentos. Este processo criará uma nova
geração de carros, que passará pelos mesmos testes, e o processo se repetirá.

Deste modo, espera-se atingir um estado de otimização em que se faz
possível o alcance da trajetória especificada sem colisões, e preferivelmente em
um caminho próximo do mais otimizado.

Apesar da elevada facilidade introduzida com esse método, a perda de
flexibilidade na utilização pode ser algo que geraria necessidade de um sistema
mais robusto como o de visão computacional. Um outro contraponto seria a
necessidade de um ambiente mais controlado, para certificar um funcionamento
correto dos sensores no ambiente e nos possíveis obstáculos. Mesmo com tais
observações, existe margem para crer que o método proposto se faz uma
alternativa interessante em aplicações em que as condições de funcionamento
podem ser mapeadas previamente.

O objetivo desta pesquisa se dá em explorar de maneira prática o funcionamento do
algoritmo de Neuroevolução de Topologias Aumentantes, de modo que seja possível 
entender seu comportamento quando aplicado a um problema prático, o qual se baseia
em cenários existentes. Este, porém, não vem a ser implementado em um ambiente real,
e sim dentro de uma simulação, de modo a compará-lo com soluções similares de 
inteligência artificial a fins de se encontrar explicações para sua forma de agir.